\documentclass[12pt]{report}

% ===========================
% Packages
% ===========================
\usepackage{amsmath, amssymb, amsfonts}
\usepackage{graphicx}
\usepackage{hyperref}
\usepackage{geometry}
\usepackage{booktabs}
\geometry{margin=1in}

% Remove paragraph indentation
\setlength{\parindent}{0pt}

% Roman numerals for sections, roman for subsections
\renewcommand\thesection{\Roman{section}.}
\renewcommand\thesubsection{\roman{subsection}.}

\begin{document}

% ===========================
% Title Page
% ===========================
\title{\textbf{Tesla Autopilot Controller}\\[4pt]
\large Gain Design Report - EML4312}
\author{Pablo F. Coral}
\date{December 2nd, 2025}
\maketitle

\tableofcontents
\newpage

% ===========================
% Project Summary
% ===========================
\section{Project Summary}

This report documents the systematic design of a cascaded control system that performs
autonomous lane-change maneuvers for a Tesla Model~S cruising at 70~mph. The autopilot
uses a PID controller for longitudinal control (x-position, parallel to the lane) and a cascaded,
proportional-only controller for lateral control (y-position, perpendicular to the current lane and pointing to the next lane). The design methodology consisted of two phases:

\begin{enumerate}
    \item \textbf{Control theory} to obtain initial gains using
    standard second-order approximations and dominant pole methods.
    \item \textbf{Optimization} to refine the gains and
    satisfy tight rise-time and settling-time specifications simultaneously.
\end{enumerate}

All eight project specifications on rise time, settling time, overshoot, and
steady-state error were satisfied by the final controller.

\subsection{Rapid iteration through Python integraton}
The gain design process was accelerated by utilizing two key python helpers: 

\begin{enumerate}
    \item A \textbf{verification script} that evaluated full system response given a set of controller gains using the given plant functions and the Scipy.signal library\footnote{The Python \emph{control} library was outaded and did not work properly with \emph{NumPy}}, which allowed for quick evaluation of performance metrics without needing to run MATLAB and perform hand-calculations each time.
    \item A \textbf{numerical optimization script} that implemented a random search algorithm with local refinement to fine-tune the gains chosen through controls-based design, which automated the otherwise tedious manual tuning process.
\end{enumerate}

\subsection{Project Resources and Reference Material}
An Appendix is included at the end of this report, containing hand calculations that provided the mathematical basis for transfer function derivations, and initial gain design.

\vspace{0.3cm}
In addition to this, the helper Python scripts, and final .m file as well as other useful documentation can be found in the github repository below:

\vspace{0.3cm}
\begin{center}
\url{https://github.com/PabloCoralDev/controls_project_fall2025}
\end{center}
\vspace{0.3cm}

\newpage

% ===========================
% I. System Models and Specifications
% ===========================
\section{System Models and Specifications}

\subsection{X-Position (Longitudinal) Plant}

From the block diagram (derivation shown in Appendix), the closed-loop transfer function for the X-position control system is shown below. 
For initial analysis, only proportional control is considered for simplification ($K_i = K_d = 0$):
\begin{equation*}
    T_x(s) = \frac{5K_p}{s^2 + 5s + 5K_p}.
\end{equation*}

Proportional control better reflects the intrinsic properties of the system dynamics.
Integral and Derivative control can be added to influence specific system properties (overshoot, steady-state error) later in the design process.

\vspace{0.3cm}

\noindent{The performance requirements for the required $22$~ft step in $x$ are:}
\begin{itemize}
    \item Rise time:\quad $0.8 < t_r < 1.1$~s,
    \item Settling time:\quad $0.8 < t_s < 1.3$~s,
    \item Percent overshoot:\quad $PO < 20\%$,
    \item Steady-state error:\quad $e_{ss} < 1$~ft.
\end{itemize}

\subsection{Y-Position (Lateral) Plant Control}

The lateral autopilot uses two coupled loops. We can, once again, begin with proportional-only control for simplification:
\begin{itemize}
    \item \textbf{Inner loop:} $K_\phi$ controls turn angle $\phi$
    \item \textbf{Outer loop:} $K_y$ controls lateral position $y$
\end{itemize}

From the block diagram (derivation also shown in Appendix), the closed-loop transfer function is:
\begin{equation*}
    T_y(s) = \frac{100 K_y K_\phi V_o}{s^2 + (100 + 100K_\phi)s + 100 K_y K_\phi V_o},
\end{equation*}
\vspace{0.3cm}
where $V_o = 102.69$~ft/s (70 mph).

\noindent{The performance requirements for the required $12$~ft step in $y$ are:}
\begin{itemize}
    \item Rise time:\quad $2.5 < t_r < 4.0$~s,
    \item Settling time:\quad $2.5 < t_s < 4.5$~s,
    \item Percent overshoot:\quad $PO < 10\%$,
    \item Steady-state error:\quad $e_{ss} = 0$~ft (exact lane centering).
\end{itemize}

\newpage

% ===========================
% II. Initial Gain Design Using Control Theory
% ===========================
\section{Initial Gain Design Using Control Theory}

\subsection{X-Position PID Controller}

The X-position controller uses a PID structure with initial conditions $K_i = K_d = 0$:
\begin{equation*}
    K_x(s) = K_p + \frac{K_i}{s} + K_d s.
\end{equation*}

Initial gain selection followed an iterative approach:

%@CLAUDE: Reference fig1 after 'yielded a great initial value' below. (fig1 is in appendix)
\begin{enumerate}
    \item Starting with $K_p = 1.5432$ (mathematical process in Appendix, page A-2) yielded a great initial value , violating only the settling time constraing by a few miliseconds. Any addition of $K_d$ at this point however, resulted in too slow of a rise time.
    \item Thus, in preparation to add derivative control for settling time improvement, $K_p$ was increased to the next closest nominal value of $K_p=1.75$.
    \item Derivative gain was added at 10\% of $K_p$\footnote{10\% of $K_p$ was chosen as a reasonable initial condition to introduce damping without destabilizing the system, while allowing for subsequent iterative refinement of $K_i$ and $K_d$.} to improve settling time, 
    resulting in $K_d \approx 0.1 K_p = 0.1(1.75) = 0.175$.
\end{enumerate}

%@CLAUDE: Reference fig2 after 'to satisfy all specifications' below. (fig2 is in appendix)
The initial gains obtained using controls theory were: $[K_p, K_i, K_d] = [1.75, 0.0, 0.175]$, which resulted in acceptable parameters except for the still-remaining violation of settling time; More iterations will be required to find an appropriate combination of 
$K_i$ and $K_d$ to satisfy all specifications, which can be done using the Python otpimization helper.

\begin{figure}[h!]
    \centering
    \includegraphics[width=0.85\textwidth]{latex_images/img1.png}
    \vspace{-0.5cm}
    \begin{center}
    \small\textit{Figure 1: X-Position Controller Performance with Initial Gains}
    \end{center}
\end{figure}

\newpage
\subsection{Y-Position Controller}

The Y-position controller uses two proportional gains in a coupled structure. From the closed-loop transfer function derived earlier, the denominator yields the characteristic equation:
\begin{equation*}
    s^2 + (100 + 100K_\phi)s + 100 K_y K_\phi V_o = 0.
\end{equation*}

\noindent{Comparing with the standard second-order form:}
\begin{equation*}
    s^2 + 2\zeta \omega_n s + \omega_n^2 = 0,
\end{equation*}
we can relate the gains to the natural frequency and damping ratio:
\begin{align*}
    \omega_n^2 &= 100 K_y K_\phi V_o, \\
    2\zeta \omega_n &= 100 + 100K_\phi.
\end{align*}

\noindent\textbf{Initial Gain Calculation:}

Targeting a rise time at the midpoint of the specification range ($t_r \approx 3.25$~s) and using the approximation $t_r \approx 1.8/\omega_n$:
\begin{equation*}
    \omega_n \approx \frac{1.8}{3.25} \approx 0.55~\text{rad/s}.
\end{equation*}

Since we have 4 unknowns ($K_\phi$, $K_y$, $\zeta$, $\omega_n$) but only 3 equations, we need to introduce an additional constraint.
We can do so by calculating the required damping ratio at the overshoot limit of 10\%, as specified by the project requirements:
\begin{equation*}
    PO = 100 e^{-\zeta\pi/\sqrt{1-\zeta^2}} = 10\%
    \quad\Rightarrow\quad
    \zeta \approx 0.59.
\end{equation*}

Using $\omega_n = 0.55$ and $\zeta = 0.59$:
\begin{align*}
    100 + 100K_\phi &= 2(0.59)(0.55) = 0.65
    \quad\Rightarrow\quad K_\phi \approx -0.99, \\
    100 K_y K_\phi V_o &= (0.55)^2 = 0.30
    \quad\Rightarrow\quad K_y \approx -0.00003.
\end{align*}

\noindent\textbf{Manual Tuning:}

Testing these gains showed the  y-position changed too slowly due to the extremely small magnitude of $K_y$. Increasing to $K_y = -0.0001$ (the next closest order of magnitude) provided much better results, but resulted in overshoot; Since further decreasing
$K_y$ would significantly slow the response, it was determined that this overshoot came as a result of an overly aggressive turn angle.

\vspace{0.3cm}

Thus, $K_\phi$ was adjusted in increments of 0.05 from $-1.0$ until $-0.980$. At $K_\phi = -0.980$ the response was slightly slow, and at $K_\phi = -0.985$ the response was slightly too fast, so a midpoint of $K_\phi = -0.9825$ was selected.

\vspace{0.3cm}

\noindent{Therefore, the initial gains for the y-position loop from controls theory were:}
\begin{align*}
    K_{\phi,\text{initial}} &= -0.9825, \\
    K_{y,\text{initial}} &= -0.0001.
\end{align*}

\newpage
Which yielded ideal performance across all project specifications, with no violations:

\begin{figure}[h!]
    \centering
    \includegraphics[width=0.85\textwidth]{latex_images/img2.png}
    \vspace{-0.5cm}
    \begin{center}
    \small\textit{Figure 2: Y-Position Controller Performance with Initial Gains}
    \end{center}
\end{figure}


These did not necessitate integral nor derivative control due to the overdamped nature of the system and the pre-existing
integrator in the system, $\frac{V_0}{s}$, which ensured near-zero steady-state error.\footnote{The negative signs arise from the sign conventions in the feedback loops; the magnitudes indicate the controller strengths.}

\newpage

% ===========================
% III. Numerical Optimization
% ===========================
\section{Numerical Optimization}

\subsection{Motivation}

The gains obtained from controls theory placed the system close to meeting all specifications, but slight violations to the project constraints remained. At this point, the bulk of the controls-based work was done, and Numerical optimization 
was employed to fine-tune the gains and satisfy all eight specifications simultaneously.

\subsection{Optimization Algorithm}
Initially, a basic gradient-descent method was attempted; the current gains were sitting at a local minimum however, and the algorithm did not succeed. Thus, a simpler random search method was implemented,
and with the assistance of a Codex agent, a local perturbation refinement step was implemented.

\vspace{0.3cm}
In short, the algorithm minimizes a cost function that penalizes violations of the eight project specifications:
\begin{equation*}
    J = \sum_i w_i \cdot \text{(violation of spec}_i)^2,
\end{equation*}
where $w_i$ are penalty weights, and violations are computed as the squared distance from specification bounds. The optimization code can be found in the following GitHub repository:

\vspace{0.3cm}
\begin{center}
\url{https://github.com/PabloCoralDev/controls_project_fall2025}
\end{center}
\vspace{0.3cm} 

\noindent\textbf{Basic algorithm Steps:}
\begin{enumerate}
    \item \textbf{Random Search:} Generate 500 random gain combinations within bounded search ranges.
    \item \textbf{Evaluation:} For each candidate, simulate the closed-loop system using scipy.signal, and compute all performance metrics.
    \item \textbf{Cost Calculation:} Penalize any metric outside its specification window.
    \item \textbf{Local Refinement:} Around the best candidates, perform small perturbations to further reduce cost.
    \item \textbf{Convergence:} Stop when a gain set satisfies all specifications (cost = 0).
\end{enumerate}

The bounded search ranges were chosen to be within reasonable distance to the initial gains:
\begin{align*}
    1.0 &\leq K_{p,x} \leq 6.0, \\
    0.05 &\leq K_{i,x} \leq 0.30, \\
    0.0 &\leq K_{d,x} \leq 0.50, \\
    0.4 &\leq K_\phi \leq 1.0, \\
    0.015 &\leq K_y \leq 0.035.
\end{align*}

\subsection{Optimized Gains}

The optimization converged to the following final gains:
\begin{gather*}
    K_{p,x} = 2.284, \\
    K_{i,x} = 0.079, \\
    K_{d,x} = 0.234, \\
    K_\phi = -0.984, \\
    K_y = -0.0001.
\end{gather*}

\underline{Key observations}: 
\begin{enumerate}
    %@ CLAUDE: reference fig3 and fig4 here (since these contain the response curve for the final tuned values). (fig3 is x-pos, fig4 is y-pos, and they are also in latex_images folder)
    \item X-position proportional gain increased from its initial value ($K_p$: 1.75 $\rightarrow$ 2.284), improving rise time and settling time performance. $K_d$ also increased proportionally to maintain damping. (new $K_d$ is now roughly 10\% of new $K_p$).
    \item A small integral gain ($K_i = 0.079$) was introduced to eliminate steady-state error.
    \item Y-position gains required minimal adjustment ($K_\phi$: $-0.9825 \rightarrow -0.984$), confirming the purely controls-based design was effective.
\end{enumerate}

\newpage

% ===========================
% IV. Final Performance
% ===========================
\section{Final Performance}

\subsection{Optimized Results}

Table~\ref{tab:final-performance} below summarizes the final, optimized performance metrics vs. specifications.


\begin{table}[h!]
    \centering
    \caption{Final Performance vs.\ Specifications}
    \label{tab:final-performance}
    \begin{tabular}{llllc}
        \toprule
        Control & Metric & Result & Spec & Satisfied? (Y/N) \\
        \midrule
        X-pos &
        $t_r$ (s) & $0.832$ & $0.8$--$1.1$ & Y \\
        \space &
        $t_s$ (s) & $1.292$ & $0.8$--$1.3$ & Y \\
        \space &
        $PO$ (\%) & $0.41$ & $< 20$ & Y \\
        \space &
        $e_{ss}$ (ft) & $0.250$ & $< 1$ & Y \\
        \midrule
        Y-pos &
        $t_r$ (s) & $2.659$ & $2.5$--$4.0$ & Y \\
        \space &
        $t_s$ (s) & $4.103$ & $2.5$--$4.5$ & Y \\
        \space &
        $PO$ (\%) & $1.03$ & $< 10$ & Y \\
        \space &
        $e_{ss}$ (ft) & $0.0058$ & $< 0.01 $ & Y \\
        \bottomrule
    \end{tabular}
\end{table}

All specifications are satisfied. The tight tolerances on rise and settling times are properly met, demonstrating the effectiveness of combining control theory with numerical optimization.

\subsection{Initial vs.\ Optimized Gains}

For the X-position loop:
\begin{align*}
    K_p &: 1.75 \;\rightarrow\; 2.284, \\
    K_i &: 0.0 \;\rightarrow\; 0.079 \quad (\text{new gain}), \\
    K_d &: 0.175 \;\rightarrow\; 0.234.
\end{align*}
\textit{Note: Derivative control remains approximately 10\% of proportional gain, maintaining the damping characteristics established during initial design.}

\vspace{0.3cm}
For the Y-position loops:
\begin{align*}
    K_\phi &: -0.9825 \;\rightarrow\; -0.984, \\
    K_y &: -0.0001 \;\rightarrow\; -0.0001 \quad (\text{unchanged}).
\end{align*}

The controls-based design phase provided exceptional initial estimates, especially for the Y-position controller which required minimal adjustment. The X-position gains were increased by roughly 30\% to meet both the faster rise and settling time requirements, and a small integral term was added to improve steady-state accuracy.

\newpage

% ===========================
% V. Conclusions
% ===========================
\section{Conclusions}

\subsection{Design Methodology Summary}

The gain design followed a two-phase methodology:
\begin{enumerate}
    \item \textbf{Controls Theory Phase:}
    \begin{enumerate}
        \item[a.] Derived closed-loop transfer functions from block diagrams
        \item[b.] Used second-order system characteristics to obtain initial gain estimates
        \item[c.] Iteratively tested and manually adjusted gains based on simulation results
    \end{enumerate}
    \item \textbf{Numerical Optimization Phase:}
    \begin{enumerate}
        \item[a.] Implemented a random search algorithm with local refinement
        \item[b.] Minimized a cost function penalizing specification violations
        \item[c.] Fine-tuned all five gains to simultaneously satisfy all project requirements
    \end{enumerate}
\end{enumerate}

\iffalse
\subsection{Key Findings}

\begin{enumerate}
    \item Classical controls theory provided strong initial estimates, placing gains within 30\% of optimal values.
    \item The Y-position cascaded controller required minimal optimization, demonstrating the effectiveness of second-order system analysis.
    \item Numerical optimization was essential for satisfying tight timing windows that were difficult to achieve through manual tuning alone.
    \item The combination of analytical design and computational refinement proved more efficient than either approach alone.
\end{enumerate}

\fi
\subsection{Final Controller Performance}
The final controller is able to exectue a safe lane change, respecting all of the project requirements. Minimum overshoot across both $x$ and $y$ is maintained
and steady-state error is minimal. Therefore, it can be concluded that the controller is effective and safe for the target application.

\vspace{0.3cm}

In addition, it can be concluded that for fine-tuning controllers with multiple variables and tight design constraints, numeriacl analysis and optimization tools, and even Machine Learning methodolgies
may be useful, or even essential in the future, for safe and effective controller design; They cannot however, override the need for a clear understanding of controls theory and system dynamics as these will
provide a basis (initial, 'rough' gains) for which to build the refined controller gain on.

% @CLAUDE: insert URL to youtube video for project demo here: https://youtu.be/c3Om5fDqyMg

\iffalse
\newpage

% ===========================
% Appendix: Formulas Used
% ===========================
\section*{Appendix: Formulas Used}
\addcontentsline{toc}{section}{Appendix: Formulas Used}

\subsection*{Second-Order System Approximations}

Given
\begin{equation*}
    s^2 + 2\zeta \omega_n s + \omega_n^2 = 0,
\end{equation*}
common approximations are:
\begin{align*}
    t_r &\approx \frac{1.8}{\omega_n}
    && \text{(rise time, $0.5\lesssim\zeta\lesssim0.7$)}, \\
    t_s &\approx \frac{4}{\zeta \omega_n}
    && \text{(settling time, 2\% criterion)}, \\
    PO &= 100 e^{-\zeta\pi/\sqrt{1-\zeta^2}}
    && \text{(percent overshoot)}.
\end{align*}

\subsection*{Overdamped Dominant Pole Method}

For a real dominant pole $s_{\text{dom}}$ (heavily overdamped systems),
\begin{align*}
    t_r &\approx \frac{2.2}{|s_{\text{dom}}|}, \\
    t_s &\approx \frac{4}{|s_{\text{dom}}|}, \\
    PO &\approx 0\%.
\end{align*}

\subsection*{Steady-State Error for Type-1 Systems}

For a unity-feedback system with loop transfer function $L(s)$:
\begin{equation*}
    T(s) = \frac{L(s)}{1+L(s)}.
\end{equation*}
For a step input of amplitude $A$:
\begin{equation*}
    y_{ss} = \lim_{s\to0} s T(s)\frac{A}{s} = A T(0).
\end{equation*}
If $L(s)$ has one integrator (Type-1), then $T(0) = 1$ and the steady-state
error for a step input is zero.

THIS SECTION ABOVE IS PROBABLY OVERKILL! DO NOT MODIFY OR DELETE JUST LEAVING HERE FOR NOW.
\fi

\newpage

% ===========================
% References
% ===========================
\section{References}

\noindent
[1] K.~J.~\AA str\"om and T.~H\"agglund, ``PID Controllers: Theory, Design, and Tuning,'' 2nd~ed. Research Triangle Park, NC: Instrument Society of America, 1995.

\vspace{0.3cm}
\noindent
[2] R.~C.~Dorf and R.~H.~Bishop, \emph{Modern Control Systems}, 13th~ed. Pearson, 2017.

\vspace{0.3cm}
\noindent
[3] G.~F.~Franklin, J.~D.~Powell, and A.~Emami-Naeini, \emph{Feedback Control of Dynamic Systems}, 8th~ed. Pearson, 2019.

\vspace{0.3cm}
\noindent
[4] K.~Ogata, \emph{Modern Control Engineering}, 5th~ed. Upper Saddle River, NJ: Prentice Hall, 2010.

\vspace{0.3cm}
\noindent
[5] ``EML4312 Project 01 -- Tesla Model~S Autopilot,'' Project Specification, University of Florida, 2025.

\newpage

% ===========================
% Appendix
% ===========================
\section{Appendix}

Figures can be found in this page. Hand calculations and derivations for transfer functions and initial gain design are included below this pageas an appended PDF of scanned files.

%@CLAUDE: Add fig1, fig2, fig3... png's from latex_images folder here in order.

\end{document}