\documentclass[12pt]{report}

% ===========================
% Packages
% ===========================
\usepackage{amsmath, amssymb, amsfonts}
\usepackage{graphicx}
\usepackage{hyperref}
\usepackage{geometry}
\usepackage{booktabs}
\geometry{margin=1in}

% Roman numerals for sections, roman for subsections
\renewcommand\thesection{\Roman{section}.}
\renewcommand\thesubsection{\roman{subsection}.}

\begin{document}

% ===========================
% Title Page
% ===========================
\title{Tesla Autopilot Controller\\[4pt]
\large Gain Design Report - EML4312}
\author{Pablo F. Coral}
\date{December 1st, 2025}
\maketitle

\tableofcontents
\newpage

% ===========================
% Executive Summary
% ===========================
\chapter*{Project Summary}
\addcontentsline{toc}{chapter}{Executive Summary}

This report documents the systematic design of a cascaded control system that performs
autonomous lane-change maneuvers for a Tesla Model~S cruising at 70~mph. The autopilot
uses a PID controller for longitudinal control (x-position, parallel to the lane) and a cascaded,
proportional-only controller for lateral control (y-position, perpendicular to the current lane and pointing to the next lane). The design methodology consisted of two phases:

\begin{enumerate}
    \item \textbf{Control theory} to obtain initial gains using
    standard second-order approximations and dominant pole methods.
    \item \textbf{Agentic-assisted optimization} to refine the gains and
    satisfy tight rise-time and settling-time specifications simultaneously.
\end{enumerate}

All eight project specifications on rise time, settling time, overshoot, and
steady-state error were satisfied by the final controller.

\newpage

% ===========================
% I. System Models and Specifications
% ===========================
\section{System Models and Specifications}

\subsection{X-Position (Longitudinal) Plant}

From the block diagram (derivation shown in Appendix), the closed-loop transfer function for the X-position control system is shown below. 
For initial analysis, only proportional control is considered for simplification ($K_i = 0, K_d = 0$):
\begin{equation*}
    T_x(s) = \frac{5K_p}{s^2 + 5s + 5K_p}.
\end{equation*}

Proportional control better reflects the intrinsic properties of the system dynamics (plant), thus simpmlifying initial gain design. 
Integral and Derivative gains can be added to influence specific system properties (overshoot, steady-state error) later in the design process.

% @CLAUDE : please add a space here!
\noindent{The performance requirements for a $22$~ft step in $x$ are:}
\begin{itemize}
    \item Rise time:\quad $0.8 < t_r < 1.1$~s,
    \item Settling time:\quad $0.8 < t_s < 1.3$~s,
    \item Percent overshoot:\quad $PO < 20\%$,
    \item Steady-state error:\quad $e_{ss} < 1$~ft.
\end{itemize}

\subsection{Y-Position (Lateral) Plant with Cascaded Control}

The lateral autopilot uses two coupled loops. We can, once again, also begin with proportional-only control for simplification:
\begin{itemize}
    \item \textbf{Inner loop:} $K_\phi$ controls turn angle $\phi$
    \item \textbf{Outer loop:} $K_y$ controls lateral position $y$
\end{itemize}

From the block diagram (derivation also shown in Appendix), the closed-loop transfer function is:
\begin{equation*}
    T_y(s) = \frac{y}{y_c} = \frac{100 K_y K_\phi V_o}{s^2 + (100 + 100K_\phi)s + 100 K_y K_\phi V_o},
\end{equation*}
where $V_o = 102.69$~ft/s (70 mph).

\noindent{The performance requirements for a $12$~ft step in $y$ are:}
\begin{itemize}
    \item Rise time:\quad $2.5 < t_r < 4.0$~s,
    \item Settling time:\quad $2.5 < t_s < 4.5$~s,
    \item Percent overshoot:\quad $PO < 10\%$,
    \item Steady-state error:\quad $e_{ss} = 0$~ft (exact lane centering).
\end{itemize}

\newpage

% ===========================
% II. Initial Gain Design Using Control Theory
% ===========================
\section{Initial Gain Design Using Control Theory}

\subsection{X-Position PID Controller}

The X-position controller uses a PID structure,

\begin{equation*}
    K_x(s) = K_p + \frac{K_i}{s} + K_d s.
\end{equation*}

Starting with \begin{equation*} K_i = K_d = 0. \end{equation*}


Initial gain selection followed an iterative approach:

% @CLAUDE : Make the stuff below an ordered list! the 'step' thing looks crammed.

\noindent\textbf{Step 1:} Starting with $K_p = 0.065$ (mathematical process in Appendix) yielded a rise time near 1.0~s but was too slow overall, with excessive steady-state error.

\noindent\textbf{Step 2:} Increasing to $K_p = 1.0$ improved response speed but still violated rise and settling time requirements.

\noindent\textbf{Step 3:} Further increasing to $K_p = 1.5$ achieved a rise time near the target. However, the system was not settling fast enough, indicating insufficient damping.

% @CLAUDE : Add a =~ (approximately) sign in the equation below:
\noindent\textbf{Step 4:} Following the Ziegler-Nichols tuning rule ($K_i K_d  0.1K_p$)derivative gain was added at approximately 10\% of $K_p$:
\begin{equation*}
    K_d \approx 0.1 K_p = 0.1(1.5) = 0.15.
\end{equation*}
Derivative control increases the damping ratio and rotates the root locus branches leftward, improving settling time.


\noindent\textbf{Step 5:} Hand-tuning between $K_p = 1.5$ and $K_p = 2.5$showed that $K_p$ to 1.75 yielded the best rise time performance. 
Adjusting the derivative gain proportionally to $K_d = 0.175$ maintained the damping ratio., and prevented a violoation of the overshoot constraint.

% @CLAUDE: center the equations below! They're messy at the moment
\noindent{The initial gains obtained using controls theory were:}
\begin{align*}
    K_{p,\text{initial}} &= 1.75, 
    K_{i,\text{initial}} &= 0.0 
    K_{d,\text{initial}} &= 0.175.
\end{align*}

\subsection{Y-Position Controller}

% @CLAUDE: improve the 'denmoinator of T_y' part below. It's messy.
The Y-position controller uses two proportional gains in a coupled structure. From the closed-loop transfer function derived earlier, the characteristic equation is:
\begin{equation*}
    s^2 + (100 + 100K_\phi)s + 100 K_y K_\phi V_o = 0. (denominator of T_y)
\end{equation*}

\noindent{Comparing with the standard second-order form:}
\begin{equation*}
    s^2 + 2\zeta \omega_n s + \omega_n^2 = 0,
\end{equation*}
we can relate the gains to the natural frequency and damping ratio:
\begin{align*}
    \omega_n^2 &= 100 K_y K_\phi V_o, \\
    2\zeta \omega_n &= 100 + 100K_\phi.
\end{align*}

\noindent\textbf{Initial Gain Calculation:}

Targeting a rise time at the midpoint of the specification range ($t_r \approx 3.25$~s) and using the approximation $t_r \approx 1.8/\omega_n$:
\begin{equation*}
    \omega_n \approx \frac{1.8}{3.25} \approx 0.55~\text{rad/s}.
\end{equation*}

% @CLAUDE: check the equations below: We are not doing 'less than' constraints, but rather, I am choosing to be at the limit for overshoot, and find out zeta at overshoot = 10%; We should find out that Zeta at this conditionos is roughly 0.592
Since we have 4 unknonws ($K_\phi$, $K_y$, $\zeta$, $\omega_n$) but only 3 equations, we need to intoduce additional constraints. 
We can do so by calculating zeta at our limit conditions of 10\% overshoot, as specified by the project requirements:
\begin{equation*}
    PO = 100 e^{-\zeta\pi/\sqrt{1-\zeta^2}} < 10\%
    \quad\Rightarrow\quad
    \zeta \gtrsim 0.6.
\end{equation*}

% @CLAUDE: zeta should be about 0.592, not 0.6; Keep sig figs consistent though
Using $\omega_n = 0.55$ and $\zeta = 0.6$:
\begin{align*}
    100 + 100K_\phi &= 2(0.6)(0.55) = 0.66
    \quad\Rightarrow\quad K_\phi \approx -0.99, \\
    100 K_y K_\phi V_o &= (0.55)^2 = 0.30
    \quad\Rightarrow\quad K_y \approx -0.00003.
\end{align*}

% @CLAUDE: None of these paragraphs have spaces between them; Please add them!
\noindent\textbf{Manual Tuning:}

Testing these gains showed the  y-position changed too slowly due to the extremely small magnitude of $K_y$. Increasing to $K_y = -0.0001$ (the next closest order of mangitude) provided much better results, but resulted in overshoot; Since further decreasing
$K_y$ would significantly slow the response, it was determined that this overshoot came as a result of an overly aggressive turn angle.

Thus, $K_\phi$ was adjusted in increments of 0.05 from $-1.0$ until $-0.980$. At $K_\phi = -0.980$ the response was slightly slow, and at $K_\phi = -0.985$ the response was slightly too fast, so a midpoint of $K_\phi = -0.9825$ was selected.

\noindent{Therefore, the initial gains for the y-position loop from controls theory were:}
\begin{align*}
    K_{\phi,\text{initial}} &= -0.9825, \\
    K_{y,\text{initial}} &= -0.0001.
\end{align*}

Which did not necessitate integral nor derivative control due to the overdamped nature of the system and the pre-existing 
integrator in the system, $\frac{V_0}{s}$, which ensured near-zero steady-state error.

% @CLAUDE: Italicize the whole note, or make it into a footnote (footnote preferred)
Note: The negative signs arise from the sign conventions in the feedback loops; the magnitudes indicate the controller strengths.

\newpage

% ===========================
% III. Numerical Optimization
% ===========================
\section{Numerical Optimization}

\subsection{Motivation}

The gains obtained from controls theory placed the system close to meeting all specifications, but slight violations to the project constraints remained. At this point, the bulk of the controls-based work was done, and Numerical optimization 
was employed to fine-tune the gains and satisfy all eight specifications simultaneously.

\subsection{Optimization Algorithm}
Initially, a basic gradient-descent method was attempted; the current gains were sitting at a local minimum however, and the algorithm did not succeed. Thus, a simpler randmo search method was implemented,
and with the assistance of a Codex agent, a local perturbation refinement step was implemented.

% @CLAUDE: Don't forget to ask me for the link to the github repo later, before I finalize the project!!
In short, the algorithm minimizes a cost function that penalizes violations of the eight project specifications:
\begin{equation*}
    J = \sum_i w_i \cdot \text{(violation of spec}_i)^2,
\end{equation*}
where $w_i$ are penalty weights, and violations are computed as the squared distance from specification bounds. The optimization code can be found in the 
the following github repository: \url{} 

\noindent\textbf{Basic algorithm Steps:}
\begin{enumerate}
    \item \textbf{Random Search:} Generate 500 random gain combinations within bounded search ranges.
    \item \textbf{Evaluation:} For each candidate, simulate the closed-loop system using scipy.signal, and compute all performance metrics.
    \item \textbf{Cost Calculation:} Penalize any metric outside its specification window.
    \item \textbf{Local Refinement:} Around the best candidates, perform small perturbations to further reduce cost.
    \item \textbf{Convergence:} Stop when a gain set satisfies all specifications (cost = 0).
\end{enumerate}

% @CLAUDE: bounded search ranges should be all in a single column (I see two columns. One that has Kp, Kd, Ky and another that has Ki and Kphi). Also, do specify that Kp, Ki and Kd correspond to Kx
The bounded search ranges were:
\begin{align*}
    1.0 &\leq K_p \leq 6.0, & 0.05 &\leq K_i \leq 0.30, \\
    0.0 &\leq K_d \leq 0.50, & 0.4 &\leq K_\phi \leq 1.0, \\
    0.015 &\leq K_y \leq 0.035.
\end{align*}

\subsection{Optimized Gains}

% @CLAUDE: Same thing as before, the equations below should be centered in a single column!
The optimization converged to the following final gains:
\begin{align*}
    K_p &= 2.284, &
    K_i &= 0.079, &
    K_d &= 0.234, \\[4pt]
    K_\phi &= -0.984, &
    K_y &= -0.0001.
\end{align*}

% @CLAUDE: Underline 'Key Observations'
Key observations: 
\begin{enumerate}
    \item X-position gains increased from the initial values (Kp: 1.75 $\rightarrow$ 2.284), improving rise and settling time performance.
    \item A small integral gain ($K_i = 0.079$) was introduced to eliminate steady-state error.
    \item Y-position gains required minimal adjustment ($K_\phi$: $-0.9825 \rightarrow -0.984$), confirming the purely controls-based design was effective.
\end{enumerate}

\newpage

% ===========================
% IV. Final Performance
% ===========================
\section{Final Performance}

\subsection{Optimized Results}

Table~\ref{tab:final-performance} below summarizes the final, optimized performance metrics vs. specifications.


% @CLAUDE: Change 'status' to 'Satisfied? (Y/N)' and have Y/N values instead of 'satisfied'
\begin{table}[h!]
    \centering
    \caption{Final Performance vs.\ Specifications}
    \label{tab:final-performance}
    \begin{tabular}{lllll}
        \toprule
        Channel & Metric & Result & Spec & Status \\
        \midrule
        X-pos &
        $t_r$ (s) & $0.832$ & $0.8$--$1.1$ & satisfied \\
        X-pos &
        $t_s$ (s) & $1.292$ & $0.8$--$1.3$ & satisfied \\
        X-pos &
        $PO$ (\%) & $0.41$ & $< 20$ & satisfied \\
        X-pos &
        $e_{ss}$ (ft) & $0.250$ & $< 1$ & satisfied \\
        \midrule
        Y-pos &
        $t_r$ (s) & $2.659$ & $2.5$--$4.0$ & satisfied \\
        Y-pos &
        $t_s$ (s) & $4.103$ & $2.5$--$4.5$ & satisfied \\
        Y-pos &
        $PO$ (\%) & $1.03$ & $< 10$ & satisfied \\
        Y-pos &
        $e_{ss}$ (ft) & $0.0058$ & $< 0.01$ & satisfied \\
        \bottomrule
    \end{tabular}
\end{table}

All specifications are satisfied. The tight tolerances on rise and settling times are properly met, demonstrating the effectiveness of combining control theory with numerical optimization.

\subsection{Initial vs.\ Optimized Gains}

% @CLAUDE: Make my note italicized here too!
For the X-position loop:
\begin{align*}
    K_p &: 1.75 \;\rightarrow\; 2.284 \\
    K_i &: 0.0 \;\rightarrow\; 0.079 \quad (\text{new gain}), \\
    K_d &: 0.175 \;\rightarrow\; 0.234.
\end{align*}
Note: Derivative control remains within the ~10\% of proportional gain guideline established by Ziegler-Nichols.

For the Y-position loops:
\begin{align*}
    K_\phi &: -0.9825 \;\rightarrow\; -0.984, \\
    K_y &: -0.0001 \;\rightarrow\; -0.0001 \quad (\text{unchanged}).
\end{align*}

The control-theoretic design provided excellent initial estimates, particularly for the Y-position controller which required essentially no adjustment. The X-position gains were increased by roughly 30\% to meet the faster rise and settling time requirements, and a small integral term was added to improve steady-state accuracy.

\newpage

% ===========================
% V. Conclusions
% ===========================
\section{Conclusions}

\subsection{Design Methodology Summary}

% @CLAUDE: Make the steps go a., b, c, etc. (within step 1 and 2)
The gain design followed a two-phase methodology:
\begin{enumerate}
    \item \textbf{Controls Theory Phase:}
    \begin{itemize}
        \item Derived closed-loop transfer functions from block diagrams
        \item Used second-order system characteristics and standard tuning rules (Ziegler-Nichols) to obtain initial gain estimates
        \item Iteratively tested and manually adjusted gains based on simulation results
    \end{itemize}
    \item \textbf{Numerical Optimization Phase:}
    \begin{itemize}
        \item Implemented a random search algorithm with local refinement
        \item Minimized a cost function penalizing specification violations
        \item Fine-tuned all five gains to simultaneously satisfy all eight project requirements
    \end{itemize}
\end{enumerate}

\subsection{Key Findings}

\begin{enumerate}
    \item Classical controls theory provided strong initial estimates, placing gains within 30\% of optimal values.
    \item The Y-position cascaded controller required minimal optimization, demonstrating the effectiveness of second-order system analysis.
    \item Numerical optimization was essential for satisfying tight timing windows that were difficult to achieve through manual tuning alone.
    \item The combination of analytical design and computational refinement proved more efficient than either approach alone.
\end{enumerate}

\subsection{Final Controller Performance}

The final autopilot controller successfully:
\begin{itemize}
    \item Executes a safe lane change maneuver in under 4.5 seconds
    \item Maintains minimal overshoot (< 1.5\%) in both longitudinal and lateral directions
    \item Achieves accurate steady-state tracking (< 0.01 ft lateral error, < 0.25 ft longitudinal error)
    \item Meets all eight project specifications simultaneously
\end{itemize}

\iffalse
\newpage

% ===========================
% Appendix: Formulas Used
% ===========================
\section*{Appendix: Formulas Used}
\addcontentsline{toc}{section}{Appendix: Formulas Used}

\subsection*{Second-Order System Approximations}

Given
\begin{equation*}
    s^2 + 2\zeta \omega_n s + \omega_n^2 = 0,
\end{equation*}
common approximations are:
\begin{align*}
    t_r &\approx \frac{1.8}{\omega_n}
    && \text{(rise time, $0.5\lesssim\zeta\lesssim0.7$)}, \\
    t_s &\approx \frac{4}{\zeta \omega_n}
    && \text{(settling time, 2\% criterion)}, \\
    PO &= 100 e^{-\zeta\pi/\sqrt{1-\zeta^2}}
    && \text{(percent overshoot)}.
\end{align*}

\subsection*{Overdamped Dominant Pole Method}

For a real dominant pole $s_{\text{dom}}$ (heavily overdamped systems),
\begin{align*}
    t_r &\approx \frac{2.2}{|s_{\text{dom}}|}, \\
    t_s &\approx \frac{4}{|s_{\text{dom}}|}, \\
    PO &\approx 0\%.
\end{align*}

\subsection*{Steady-State Error for Type-1 Systems}

For a unity-feedback system with loop transfer function $L(s)$:
\begin{equation*}
    T(s) = \frac{L(s)}{1+L(s)}.
\end{equation*}
For a step input of amplitude $A$:
\begin{equation*}
    y_{ss} = \lim_{s\to0} s T(s)\frac{A}{s} = A T(0).
\end{equation*}
If $L(s)$ has one integrator (Type-1), then $T(0) = 1$ and the steady-state
error for a step input is zero.

THIS SECTION ABOVE IS PROBABLY OVERKILL! DO NOT MODIFY OR DELETE JUST LEAVING HERE FOR NOW.
\fi

\newpage

% ===========================
% References
% ===========================
\begin{thebibliography}{9}
    \bibitem{astrom_pid}
    K.~J.~\AA str\"om and T.~H\"agglund,
    ``PID Controllers: Theory, Design, and Tuning,'' 2nd~ed.
    Research Triangle Park, NC: Instrument Society of America, 1995.

    \bibitem{dorf}
    R.~C.~Dorf and R.~H.~Bishop,
    \emph{Modern Control Systems}, 13th~ed.
    Pearson, 2017.

    \bibitem{franklin}
    G.~F.~Franklin, J.~D.~Powell, and A.~Emami-Naeini,
    \emph{Feedback Control of Dynamic Systems}, 8th~ed.
    Pearson, 2019.

    \bibitem{ogata}
    K.~Ogata,
    \emph{Modern Control Engineering}, 5th~ed.
    Upper Saddle River, NJ: Prentice Hall, 2010.

    \bibitem{project}
    ``EML4312 Project 01 -- Tesla Model~S Autopilot,''
    Project Specification, University of Florida, 2025.
\end{thebibliography}

\end{document}